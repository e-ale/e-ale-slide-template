\section{Lab 4}

\begin{frame}
   {Userspace Input Driver}

   In Lab 4, we will accomplish the following:
	\begin{itemize}
		\item
			Learn about the kernel uinput module, python-uinput, python-libiio, and python-periphery
		\item
			Create a userspace joystick driver
		\item
			Test the joystick driver
	\end{itemize}
\end{frame}

\begin{frame}
    {uinput drivers}

	Linux userspace input driver provides methods to:
\begin{itemize}
	\item
		Register input devices with the kernel
	\item
		Register types of events the userspace driver may generate (and ranges if applicable. e.g. BTN, KEY, ABS
	\item
		Trigger events from userspace

	The uinput driver interface and C APIs are described in detail at \url{https://www.kernel.org/doc/html/v5.0/input/uinput.html}
\end{itemize}
\end{frame}

\begin{frame}
    {Write a uinput joystick driver in Python}

	\begin{itemize}
		\item
	\textbf{python-uinput} provides Python support for uinput drivers.
		\item
	\textbf{python-libiio} provides Python support for IIO devices.
		\item
	\textbf{python-periphery} provides access to various embedded I/O facilities including GPIO.
	\end{itemize}
\end{frame}

\begin{frame}
	{Python joystick driver}
	\begin{raw}
#!/usr/bin/env python3

from periphery import GPIO
import iio
import time
import uinput

events = (
    uinput.ABS_X + (-255,256, 0, 0),
    uinput.ABS_Y + (-255,256, 0, 0),
    uinput.BTN_A,
    uinput.BTN_B,
    )

with uinput.Device(events) as device:
    contexts = iio.scan_contexts()
    uri = next(iter(contexts), None)
    ctx = iio.Context(uri)
	\end{raw}
\end{frame}

\begin{frame}
	{Python joystick driver (continued)}
	\begin{raw}
    mma8453 = ctx.find_device("mma8453")

    x = mma8453.find_channel("accel_x", False)
    y = mma8453.find_channel("accel_y", False)

    a = GPIO(45, "in")
    b = GPIO(117, "in")

    while (1):
        device.emit(uinput.ABS_X, int(x.attrs["raw"].value))
        device.emit(uinput.ABS_Y, int(y.attrs["raw"].value))
        device.emit(uinput.BTN_A, not a.read())
        device.emit(uinput.BTN_B, not b.read())
        time.sleep(.1)
	\end{raw}
	Start the driver:
	\begin{raw}
root@beaglebone:~# ./techjoy.py &
	\end{raw}

\end{frame}

\begin{frame}
{Test the joystick driver (evtest)}
	\begin{rawscriptsize}
root@beaglebone:~# evtest /dev/input/event1
Input driver version is 1.0.1
Input device ID: bus 0x19 vendor 0x0 product 0x0 version 0x0
Input device name: "ocp:joystick"
Supported events:
  Event type 0 (EV_SYN)
  Event type 1 (EV_KEY)
    Event code 304 (BTN_SOUTH)
    Event code 305 (BTN_EAST)
  Event type 3 (EV_ABS)
    Event code 0 (ABS_X)
      Value    -31
      Min     -255
      Max      256
    Event code 1 (ABS_Y)
      Value     40
      Min     -255
      Max      256
Properties:
Testing ... (interrupt to exit)
Event: time 1566259060.295970, type 3 (EV_ABS), code 0 (ABS_X), value -31
Event: time 1566259060.295970, -------------- SYN_REPORT ------------
Event: time 1566259060.547851, type 3 (EV_ABS), code 0 (ABS_X), value -29
Event: time 1566259060.547851, -------------- SYN_REPORT ------------
Event: time 1566259060.631933, type 3 (EV_ABS), code 0 (ABS_X), value -30
Event: time 1566259060.631933, type 3 (EV_ABS), code 1 (ABS_Y), value 41
.
.
.
	\end{rawscriptsize}

	Try moving the board around, what are the results?
\end{frame}

\begin{frame}
{Test the joystick driver (jstest)}

	\begin{rawscriptsize}
root@beaglebone:~# jstest /dev/input/js0 
Driver version is 2.1.0.
Joystick (python-uinput) has 2 axes (X, Y)
and 2 buttons (BtnA, BtnB).
Testing ... (interrupt to exit)
Axes:  0:   642  1:  2441 Buttons:  0:off  1:off ^C
	\end{rawscriptsize}

	Try moving the board around, what are the results?
\end{frame}

\begin{frame}
    {Play Tetris}

	\begin{itemize}
		ssh to the PocketBeagle for better performance
	\end{itemize}

	\begin{raw}
root@beaglebone:~# tar jxf $HOME/test/vitetris-0.57.tar.gz
root@beaglebone:~# cd vitetris-0.57
root@beaglebone:~# ./configure
root@beaglebone:~# make
root@beaglebone:~# ./tetris
	\end{raw}
\end{frame}
