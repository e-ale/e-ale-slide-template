\section{GPIO Subsystem}

\begin{frame}
   {Overview}

   \begin{itemize}
      \item
	      The Linux GPIO subsystem is a framework to support control of General Purpose Input/Output pins
      \item
		\url{https://www.kernel.org/doc/Documentation/gpio/gpio.txt}
      \item
	      No longer using the legacy GPIO APIs.
		   \begin{itemize}
			   \item 
				   \url{https://www.kernel.org/doc/Documentation/gpio/gpio-legacy.txt}
		   \end{itemize}
      \item
	      Prefer the descriptor-based consumer GPIO APIs.
		   \begin{itemize}
			   \item
				   \url{https://www.kernel.org/doc/Documentation/gpio/consumer.txt}
		   \end{itemize}
   \end{itemize}
\end{frame}

\begin{frame}
	{Consumer}

	\begin{itemize}
		\item
			Get a GPIO descriptor
			\begin{raw}
struct gpio_desc *devm_gpiod_get_index(struct device *dev,
				       const char *con_id,
			               enum gpiod_flags flags)
			\end{raw}
			\begin{itemize}
		\item
			\textbf{con\_id} is typically the prefix of a \textbf{Device Tree gpio(s)} property. e.g. a \textbf{power-gpio} property would require \textbf{power} for \textbf{con\_id}
					\begin{itemize}
						\item
							\url{https://www.kernel.org/doc/Documentation/gpio/board.txt}
					\end{itemize}
		\item
			\textbf{flags} are optional and can include direction and/or initial value for a GPIO. e.g. \textbf{GPIOD\_IN} for an input
			\end{itemize}
		\item
			Get a GPIO value (0 for low, nonzero for high)
			\begin{raw}
int gpiod_get_value(const struct gpio_desc *desc);
			\end{raw}
	\end{itemize}
\end{frame}
