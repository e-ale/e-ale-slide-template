\section{Linux Distributions and Desktops}

\begin{frame}
   {Linux Distributions}

   \begin{itemize}
      \item
      There are many available \textbf{Linux} distributions
      (See \url{http://lwn.net/Distributions})
      \item
      This course is \textbf{distribution-flexible}
      \item
      Three main families we work with:
      \begin{itemize}
         \item
         \textbf{Red Hat Enterprise Linux 7 } (\textbf{RHEL
         7})

         (And close cousins \textbf{CentOS},
         \textbf{Fedora}, \textbf{Scientific Linux})
         \item
         \textbf{Ubuntu}

         A derivative of \textbf{Debian} (Close cousin
         \textbf{Mint Linux})
         \item
         \textbf{openSUSE}

         Closely related to \textbf{SLES} (\textbf{SUSE
            Linux Enterprise Server} or just \textbf{SUSE})
   \end{itemize}
   \item
   The exact distribution version is not critical; latest release
   and previous one or two releases should be fine.
\end{itemize}

\end{frame}

\cprotect\note{

   Broadly speaking, there are three main families of
   \textbf{Linux} distributions:
   \begin{enumerate}
      \item \textbf{Red Hat}
      \item \textbf{Debian}
      \item \textbf{SUSE}
   \end{enumerate}

   Each family has offspring and other relatives.

   For \textbf{Red Hat}, \textbf{RHEL} is the base
   commercial product, while \textbf{Fedora} is the
   community-based project which is used as a testing
   platform for future development.  \textbf{CentOS} is
   essentially a clone of \textbf{RHEL} that is freely
   available, and is actually a part of \textbf{Red Hat}
   today.  \textbf{Scientfic Linux} is an independent clone
   used in the research community and other places.

   \textbf{Debian} is the community-based project which is
   the parent of the commercial \textbf{Ubuntu}
   distribution, which piggybacks on \textbf{Debian}.
   \textbf{Mint Linux} is a derivative of \textbf{Ubuntu}
   which gets rid of some \textbf{Ubuntu} quirks, such as
   the \textbf{Unity} desktop environment.

   \textbf{openSUSE} has a similar relationship to
   \textbf{SLES} as \textbf{Fedora} does to \textbf{RHEL};
   it is a community-based version of the enterprise
   distribution, containing cutting edge features being
   tested.

   There are \textbf{Linux} distributions which are clearly
   outside of these three families, such as \textbf{GENTOO}
   and \textbf{Arch Linux}, both of which are relatively
   cutting edge and advanced, and are unlikely to be used
   by new \textbf{Linux} users.

   There are also some narrowly focused distributions used
   in \textbf{Linux} embedded devices, and in addition, the
   \textbf{Android} operating system used on mobile
   devices, is based on the \textbf{Linux} kernel, but is
   quite different in its user interface.

}


\begin{frame}
   {Keeping up to Date}
   \begin{itemize}
      \item
      The entire \textbf{Linux} ecosystem is constantly
      changing
      \item
      \textbf{Linux} is not a mono-culture, but is very diverse
      \item
      Each distribution has its own strategy for updating,
      upgrading and issuing new releases
      \item
      Not everything will work exactly right on all
      distribution versions at all times
      \item
      To make sure everything worked perfectly, on all
      distributions and versions, at all times, we would have
      to freeze content, which would rapidly go stale
      \item
      It is better to stay current than freeze content.
      \item
      Figuring out these changes and how to adapt is part of
      the \textbf{Linux} experience!
   \end{itemize}

\end{frame}

\cprotect\note{

   \textbf{Linux} offers a bewilderingly large variety of
   options and choices, both in choosing which distribution
   to run, and in configuring the distribution, including
   which software to install.

   This means it is impossible to write documentation
   (including this course) which can cover every possible
   configuration that might be encountered.

   One could, of course, choose one particular distribution
   and release and base everything on that; indeed that is
   what commercial \textbf{Linux} distributions do when
   providing training and documentation.

   However, for the \textbf{Linux Foundation}, it is
   important to support the entire ecosystem and community
   and that is not a viable option.  Besides it is just too
   limiting!

}


\begin{frame}
   {Linux Desktops}

   \begin{itemize}
      \item There are two major \textbf{Desktop} environments
      widely available in \textbf{Linux}: \textbf{GNOME} and
      \textbf{KDE}
      \item
      Most \textbf{Linux} distributions offer a choice between
      the two
      \item
      However, most have a preferred default, which is usually
      \textbf{GNOME}
      \item
      Each desktop manager also has different versions; e.g.,
      both \textbf{GNOME 2} and \textbf{GNOME 3} are widely
      used.
      \item
      This means we can not give precise instructions that
      will work in every GUI, but it is usually easy to figure
      out.
      \item
      Other desktops are also used; e.g., \textbf{XFCE} which
      is far less resource-intensive but also very comfortable
      and robust.
   \end{itemize}

\end{frame}

\cprotect\note{

   We will discuss the desktop environments in detail
   later.

   However, we just want to point out here it is impossible
   to give detailed recipes for how to do things in the
   various menus etc that are part of the \textbf{GUI}, due
   to the diversity of desktops available.

   Note that which environment is best is a very contentious
   issue, one of the \textbf{holy wars} which come up in
   the open source community on many issues.

   We will focus more on \textbf{GNOME} in this course only
   because it has a larger market share, even when it is
   hidden underneath the \textbf{Unity} interface of
   \textbf{Ubuntu}.

}


