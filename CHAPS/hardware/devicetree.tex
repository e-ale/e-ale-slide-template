\section{Device Tree}

\begin{frame}
    {Overview}

	A Device Tree defines the hardware configuration for a platform. The following resources provide details on DT.
\begin{itemize}
	\item
		Device Tree Specification
		\url{https://github.com/devicetree-org/devicetree-specification/releases/download/v0.2/devicetree-specification-v0.2.pdf}
		Register input devices with the kernel
	\item
		Kernel Device Tree Documentation
		\url{https://www.kernel.org/doc/Documentation/devicetree/}
	\item
		Device Tree for Dummies
		\url{https://elinux.org/images/f/f9/Petazzoni-device-tree-dummies_0.pdf}
\end{itemize}
\end{frame}

\begin{frame}
	{IIO provider binding}

	Documentation/devicetree/bindings/iio/iio-bindings.txt:

	\begin{raw}
==IIO providers==

Required properties:
#io-channel-cells: Number of cells in an IIO specifier; Typically 0 for nodes
                   with a single IIO output and 1 for nodes with multiple
                   IIO outputs.

Example for a simple configuration with no trigger:

        adc: voltage-sensor@35 {
                compatible = "maxim,max1139";
                reg = <0x35>;
                #io-channel-cells = <1>;
        };
	.
	.
	.
	\end{raw}
\end{frame}

\begin{frame}
	{IIO consumer binding}

	Documentation/devicetree/bindings/iio/iio-bindings.txt:
	\begin{raw}
==IIO consumers==

Required properties:
io-channels:    List of phandle and IIO specifier pairs, one pair
                for each IIO input to the device. Note: if the
                IIO provider specifies '0' for #io-channel-cells,
                then only the phandle portion of the pair will appear.

Optional properties:
io-channel-names:
                List of IIO input name strings sorted in the same
                order as the io-channels property. Consumers drivers
                will use io-channel-names to match IIO input names
                with IIO specifiers.

For example:

        device {
                io-channels = <&adc 1>, <&ref 0>;
                io-channel-names = "vcc", "vdd";
        };
	\end{raw}
\end{frame}

\begin{frame}
	{MMA8453 binding}

	Documentation/devicetree/bindings/iio/accel/mma8452.txt:
	\begin{rawscriptsize}
Freescale MMA8451Q, MMA8452Q, MMA8453Q, MMA8652FC, MMA8653FC or FXLS8471Q
triaxial accelerometer

Required properties:

  - compatible: should contain one of
    * "fsl,mma8451"
    * "fsl,mma8452"
    * "fsl,mma8453"
    * "fsl,mma8652"
    * "fsl,mma8653"
    * "fsl,fxls8471"

  - reg: the I2C address of the chip

Optional properties:

  - interrupts: interrupt mapping for GPIO IRQ

  - interrupt-names: should contain "INT1" and/or "INT2", the accelerometer's
                     interrupt line in use.

Example:

        mma8453fc@1d {
                compatible = "fsl,mma8453";
                reg = <0x1d>;
                interrupt-parent = <&gpio1>;
                interrupts = <5 0>;
                interrupt-names = "INT2";
	\end{rawscriptsize}
\end{frame}

\begin{frame}
	{Pinctrl client binding}

	Documentation/devicetree/bindings/pinctrl/pinctrl-bindings.txt:
	\begin{raw}
Required properties:
pinctrl-0:      List of phandles, each pointing at a pin configuration
                node. These referenced pin configuration nodes must be child
                nodes of the pin controller that they configure.
.
.
.
Optional properties:
pinctrl-1:      List of phandles, each pointing at a pin configuration
                node within a pin controller.
.
.
.
For example:

        /* For a client device requiring named states */
        device {
                pinctrl-names = "active", "idle";
                pinctrl-0 = <&state_0_node_a>;
                pinctrl-1 = <&state_1_node_a &state_1_node_b>;
        };
	\end{raw}
\end{frame}

\begin{frame}
	{GPIO consumer binding}

	Documentation/devicetree/bindings/gpio/gpio.txt:
	\begin{raw}
.
.
.
GPIO properties should be named "[<name>-]gpios", with <name> being the purpose
of this GPIO for the device.
.
.
.
Example of a node using GPIOs:

        node {
                enable-gpios = <&qe_pio_e 18 GPIO_ACTIVE_HIGH>;
        };

GPIO_ACTIVE_HIGH is 0, so in this example gpio-specifier is "18 0" and encodes
GPIO pin number, and GPIO flags as accepted by the "qe_pio_e" gpio-controller.
.
.
.
	\end{raw}
\end{frame}
